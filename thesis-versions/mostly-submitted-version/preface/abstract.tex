
% Thesis Abstract -----------------------------------------------------

%\begin{abstractslong}    %uncommenting this line, gives a different abstract heading
\begin{abstracts}        %this creates the heading for the abstract page
Hierarchical models and observations show that galaxy interaction and merging is of paramount importance to galaxy assembly and evolution. However, the relationship between these physical processes and the characteristics of the galaxies involved is unclear. In this thesis, we make direct constraints between the physical processes occuring in galaxy interaction - increased star formation, nuclear activity, and morphological disturbance - and the underlying parameters of galaxies - their interaction stage, stellar masses, kinematics, and orientation parameters. 

To constrain these relations, we require large samples of interacting galaxies which are representive of the full underlying parameter space. A clear signature of interaction or merger activity is through the morphological distortion of a galactic system. We search for such signatures through the entire \emph{Hubble Space Telescope} science archive with a Bayesian convolutional neural network. In total, we classify 92 million sources into interacting and non-interacting galaxies. We find 21,926 disturbed and interacting galaxy systems; the largest interacting galaxy sample morphologically classified to date. 

We use this new sample to explore the relationship between the dynamical timescale of interaction and galactic parameters. We match our new sample with the Cosmic Evolutionary Survey and find ancillary data for a mass limited sample of 3,384 interacting galaxies. We break this sample into four distinct stages aimed at capturing different parts of the dynamical timescale of interaction. We find evolution in the star formation rate with stage which cumulates in the complete disappearence of the red sequence at the merging stage. We find that the fraction of galaxies with an active galactic nuclei is constant with interaction stage, except at the point of coalescence where we find it increases from the field value of 0.12 to 0.15. By investigating the relationship between these fundamental processes and stage, we show that there is a direct relation between the interaction stage and the physical processes we observe. However, we require methods to more precisely probe the dynamical timescale of an interaction to make further constraints.

Thus, we introduce a new algorithm which will be capable of exploring this relationship. We utilise a three-body numerical simulation with a Markov-Chain Monte Carlo (MCMC) algorithm to directly map the parameter space to tidal features. We constrain the underlying parameters of a sample of 51 synthetic images representing different observed interacting systems. We are able to recover the true parameters within a confidence interval of 86.4\%. We apply this methodology to a subset of observational data, and explore the algorithms potential as well as its limitations.

The results of this thesis demonstrates the complex nature of the relationship between galaxy evolution and interaction, with larger samples and catalogues needed to constrain it. With the use of ESA Datalabs, buidling such samples is not just limited to interacting galaxies. It will allow the largest general galaxy morphology catalogues to date to be created. This, therefore, requires ever larger catalogues of ancillary data or the development of methods to efficiently create it. The introduction of a new algorithm to further investigate the relationship of interaction and its underlying parameters provides us a unique way to link interaction and evolution. The limitations of this algorithm are primarily due to the computational expense such a MCMC approach costs. We explore how this method can be developed, from bypassing the computational expense entirely with simulation based inference to new methods of optimisation with graphical processing units.

\end{abstracts}

%\end{abstractlongs}


% ---------------------------------------------------------------------- 
