\chapter{Introduction}
% Introduce Galaxy Interaction, why is it important?
In the context of large scale structure of the universe, galaxies represent a fundamental building block of matter. While the universe is full of filaments and voids, they are specifically over- (or under-) densities of galaxies which trace out the distribution of matter. To understand the distribution of matter in the universe (and further fundamental questions about it) we must understand how galaxies form and evolve \citep{2005Natur.435..629S}. A major process which is key to this understanding is that of mutual interactions between galaxies. Mutual interaction has a number of influences on galaxies, from morphological distortion, to mass transfer, to increases in star formation, to merging where the two (or more) galaxies coalesce into a single system. Our best cosmological model, $\Lambda$-Cold Dark Matter, dictates that matter - ergo, galaxies - assembled hierarchically \citep{}. Therefore, mutal interaction and galaxy merging must play an important role in the evolution of matter in the universe to the present day.

We have come to understand that interaction has multiple impacts on the evolution of galaxies. First, it leads to the distortion of the disks involved and the formation of dinstinct tidal features. Since the existence of such features being related to interaction was hypothesised to proved \citep{Toomre_72} they are the best indication that two galaxies are interacting. We have a notoriously hard time identifying interacting systems in the absence of tidal features of the galaxies involved. This is because, without adequete spectroscopic information of the systems, it is very difficult for us to tell if two systems are physically close together or actually close together by projection effects on the sky. Therefore, studies identifying interacting galaxies often focus primarily on the morphological distortion of a system.

However, there are more subtle affects due to interaction that remain poorly understood. It is widely debated about the role of interaction in nuclear ignition and enhancement in star formation of each system. While the former may not happen at all, the latter is generally accepted to occur while the specific strength and impact on the system remains highly debated. If such processes occur strongly, they can completely change the system that we are observing. Interaction has also been proved to have an impact on the gas reserves in a galaxy, often leading to the complete quenching of the system post-interaction. However, interaction has also been shown to rejuvenate the galaxies involved, and bring them out of a quenched state and into a star forming state. This is highly dependent on the specifics of the interaction. Finally, the role of environment in galaxy interaction is also a highly researched area.

The primary focus of this thesis is on the role of tidal distortion in formation of tidal features for the purposes of system identification and constraint. However, in this process, we must also keep an account of the other underlying processes occuring during interaction. Fully understanding them, and modelling them if necessary, is key to accurately modelling an interaction. The unresolved questions about galaxy interaction is long and complex. The focus of this thesis is in providing the tools to understand galaxy interaction and answer some of the fundamental questions regarding it. These questions include:

\begin{itemize}
	\item What is the relationship between star formation enhancement and interaction?
	\item What is / is there a relationship between nuclear activation and interaction?
	\item What is the relationship between the tidal features that form and the interaction?
	\item What is the relationship between the environment about a galaxy and interaction?
\end{itemize}

To fully answer these questions about galaxy interaction, we require two things: large, statistically significant, representative interacting galaxy samples and pipelines to analyse those large samples. In this thesis, detail two pipelines which achive these goals. First, I describe a pipeline that will be able to analyse large interacting galaxy samples in the context of Bayesian statistics and galactic parameterisation (chapter 2). I will describe the results we have found by applying it to a small, well constrained, interacting galaxy sample and then describe the limitations of the approach in terms of computational efficiency. In chapter 3, I will describe a pipeline for creating the largest interacting galaxy catalogue to date from combining morphological classification with novel methods of data extraction and analysis. Finally, in chapter 4 I will use the found large galaxy sample cross matched with a galaxy parameterisation catalogue and explore the above questions in the context of a large, representative sample. However, for now, it would be prudent to begin by discussing what we do understand about galaxy interaction and the relationship it has with the underlying processes occurring in a galaxy.

\section{GALAXY INTERACTION}
% Here, present examples of interacting galaxies and the context in which we're talking about them.
Initially, at the discovery of galaxies and they were assumed to be island universes \citep{Hubble paper}. The idea of two or more galaxies interacting and the effect that it would have had upon galaxy formation and evolution was thought to be minute \citep{Found this info in Chapter 8 of Binne and Tremaine. Need to find a better source!}. This assumption was not entirely without merit for the time. By calculating the chance of a galactic encounter from the galactic cross section measured via the galactic radius of stellar material and comparing it to the vast distances between galaxies, it seemed highly improbable that two systems would collide. However, with updated cosmological theories and the introduction of a model where the stellar component of galaxy is embedded in a dark matter halo \citep{Which paper introduced the idea of a galactic halo?}, the size of galaxies and their cross section increased. This dramatically increased the probability of two galaxies interacting in the universe, and with the theory that may of the tidal features found were formed by tidal interaction, a new field of study was born. 

We now understand that galaxy interaction plays an important, and fundamental role in the evolution of galaxies. As stated previously, they have multiple affects upon the systems undergoing the interaction. While the specifics of these effects lies heavily on the underlying parameters and type of interaction that is occurring, the driving force underlying them all is graviational. Therefore, the mass of the systems involved is of great importance to the resultant system we might observe. If the two systems that interact are at vastly different masses - a so-called minor interaction - then the less massive galaxy will be highly perturbed, whilst the more massive galaxy will be relatively intact. If, however, the two systems are of a relatively similar mass both systems will be highly perturbed to the point where they may even be destroyed - a so-called major interaction. A much more difficult type of interaction to observe is a micro-interaction. This is where one system has an almost insignificant mass when compared to the system it is interacting with. In this scenario, the least massive system is always almost entirely destroyed by the more massive system. Specific criteria are often placed upon the mass ratios - the mass of one galaxy when compared to the other - to define these three categories of interaction. For a major interaction the mass ratio must be close to 1:1, in a minor interaction it must be close to 1:10 and in a micro interaction it must be less than 1:10 \citep{A paper that states what the accepted merger ratios are?}. 

However, it is also known that other aspects of the interaction also influence the outcome of the final system. Another that has a direct impact on the output morphology of the system is the velocities and orientations involved in the interaction \citep{Some review paper, perhaps that states that these parameters are important}. A slower interaction with a higher impact parameter (i.e. one where the point of closest approach is small) leads to a more violent interaction with highly perturbed morphology. These slower interactions often lead to the two or more systems becoming intertwined, with multiple passages and merging being the final state of the system. This often occurs due to dynamical friction in the interaction, where heating of the two disks involved and transfer of momentum in the inelastic interaction leads to the two systems spiralling into one another. Again, such an outcome is directly related to the masses of the systems involved as well as their relative velocities and their orientations. 

Many underlying physical processes can also be started or enhanced due to interaction, and the effects of gravity upon the galaxies involved. If the systems involved are gas-rich - a so-called wet merger - then the star formation in the involved systems can be significantly enhanced. It has often been found that galaxies in an interaction or in the post-interaction state are starbursting \citep{Paper on starbursting in interaction}, with their star formation rates many times what would expected of a galaxy of their mass. This increase in star formation then leads to quenching of the system. In the same way, a system that is already quenched can undergo rejuvenation of its gas, and begin star forming after a wet merger. The increased movement in gas throughout the system can also lead to the activation of nuclear activity in the core, and completely alter what we observe of a galaxy. On the other hand, in a dry merger, no increase in star formation is expected and the discussed effects are often surpressed.

So, with these underlying processes in mind, how do they actually occur? What is happening within galaxies that leads to interaction having such a profound effect upon their observed parameters. To understand this, we must go back to first principles and explore how these processes would normally occur and then extrapolate this out the extreme conditions of interaction. I will explore how these conditions relate directly to the underlying parameters of interaction, and the observational effects each result has on a galactic system.

% For each of these sections below, I want to do a deep dive on how each process is driven. Will maybe mention simulations throughout, but we shall see?
\subsection{Morphological Distortion: Tidal Features}
% Introduction here: What do I mean by tidal distortion? What are we actually looking at?
\noindent The clearest example of two galactic systems interacting is the existence of tidal features. These tidal features are results of severe morphological distortion due to tidal forces an interaction. An excellent example which has been very highly studied is NGC 4038 / 4039, also known as the Antennae galaxy. This system is a major interacting pair, which are likely continuing into a merger, with long tidal tails extending far from the interacting cores. The first works attempting to simulate the outcomes of galaxy interaction focused on this system \citep{Works like TandT which looked at this}. The full extent of the tidal tails system was not revealed until deeper observational data became available in 2004 \citep{Citation from BandM}, and then could be accounted for in the simulators modelling.

This is a common problem with information on the tidal features which form in interaction. They often lie in the low surface brightness regime, at the very deepest depths that our telescopes can observe. This often makes investigating any system but the most massive, brightest systems difficult. A classic example of the limitations of our telescopes has been the case of NGC 5907, where the morphology of the stellar stream about it (likely created by a minor interaction) has been under significant scrutiny in recent years \citep{If this paragraph remains here, cite the different NGC 5907 papers}. With the changing morphology of the stellar stream, so to does our understanding of the dynamics of the system as well as estimates of the progenitor galaxy that formed the stellar stream. 

Thus, understanding the dynamics at play in an interaction, and the underlying physics of the different tidal features which form, is of paramount importance to understanding galaxy interaction. As an example, I will discuss a simple galaxy interaction with only two galaxies involved - a primary and a secondary - and assume that prior to the interaction they are disk galaxies. Each of these systems is composed of three parts: a stellar disk, a gas disk and the dark matter halo each disk is embedded in. Starting from a very far distance, and assuming that at such distances the galaxies can be approximated as dark matter halos and are simply attracted to each other by the force of gravity,
\begin{equation}
	F_{12} = \frac{GM_{1}M_{2}}{r_{12}^{2}}.
\end{equation}

Here, G is the gravitational constant, M$_{1}$ and M$_{2}$ are the total masses of the primary and secondary and r$_{12}$ is the distance between the two galaxies. Each galaxy has a gravitational potential about it. This is defined by
\begin{equation}
	E_{p} = -\frac{GM}{r},
\end{equation}
where r is the distance to any point around our approximated point mass of the galaxy. Therefore, in an interaction, we combine the gravitational potential energy at any point based on their distances from the two galaxies.  Hence, it immediately becomes obvious that the gravitational potential about each galaxy will become distorted in the interaction.

This is where our point-like approximation of each galaxy begins to fall apart. Galaxies are, of course, not truly point like objects but are extended objects containing many different massive objects which will affect the gravitational potential themselves. However, for our purposes here, we can think of the stellar and gas disks of the 
% Go into the physics now of a tidal interaction, what is happening gravitationally and with equations?

% What is the 'experience' with stars? What about gas?

% Finishing talking about gas here allows us to easily follow up and talk about the star formation enhancement next...

\subsection{Star Formation Enhancement} 
\subsection{Nuclear Activation}
\subsection{Quenching of Galactic Systems}
\subsection{The Role of Environment}

\section{SIMULATIONS OF GALAXY INTERACTION}
% Here, we discuss how we often simulate galaxy interaction.

\subsection{Underlying Parameters of Interaction}

\subsection{Prior Examples}
% Introduce prior works, from numerical simulations to hydrodynamical.
% Discuss their pros and cons.

\subsection{PySPAM}
% State that we are going to focus on numerical simulations (restricted N-body codes). Break down how our code (JSPAM) works.

\section{STATSTISTICALLY CONSTRAINING INTERACTION}
% Introduce what I mean by statistical constraint.

\subsection{Bayesian Statistics \& MCMC}
% Have to do a deep dive on Bayesian Statistics and such here...

\subsection{Previous Examples}
% May not exist, but previous examples of this approach?

\section{INTERACTING GALAXY IDENTIFICATION}
% Now, constraining interaction is one problem, but we don't even have that many to constrain!!

\subsection{By Citizen Scientists}
% Discuss the methods of Galaxy Zoo: Mergers, visual classification and the issues that arise.

\subsection{By Machine Learning}
% Discuss previous attempts with Machine Learning, pros and cons and where to go from here.

\subsection{Existing Spectroscopic Surveys}
% As the only thing we can use is spectroscopic information to truly categorise interacting galaxies, where could we go?


\section{LARGE SAMPLES OF INTERACTING GALAXIES}
\subsection{Created by Citizen Scientists}
\subsection{Found with Machine Learning}
\section{Thesis Outline}

\section{ANALYSIS OF GALAXY INTERACTION}
\subsection{Galactic Parameterisation}
\subsection{Observed Differences in Interacting Galaxies}
\subsection{What still needs to be learned}

\section{A PIPELINE FOR GALAXY IDENTIFICATION AND ANALYSIS}