\chapter{Conclusion}
The stated objective of this thesis was to present a viable extension of general relativity, which is free from singularities, where `viable', in this case means devoid of ghosts, tachyons or exotic matter. With this in mind, we outline the results of the present work.
\subsubsection*{Outline of Results}
We began in Chapter \ref{chap:2}, with a lengthy computation of the non-linear field equations for the IDG action \eqref{action}. First, we outlined the general methodology and introduced a number of novel techniques, which proved useful in the explicit calculation that followed. Having attained the non-linear field equations, we perturbed around both Minkowski space and de Sitter space  up to linear order, for later use in deriving the modified propagator, ghost-free conditions and defocusing condition, required to construct a viable non-singular cosmology.

Chapter \ref{chap:GF}, focused on the notion of ghosts or tachyons that may appear at the level of the propagator. First, these manifestations were defined and illustrated with a number of examples from finite extensions of GR. The form of the modified propagator was also derived for the IDG action \eqref{action}. Next, the elimination of ghosts by way of an exponential correction in the scalar propagating mode was motivated, before being put to use in a linearised regime around Minkowski space. Typically, such an exponential function in the propagator weakens the classical and quantum effects of gravity in the UV. 
%For instance, it was found, at the linearised limit, that there are no blackhole-like singularities in a static spacetime~\cite{Biswas:2011ar}, nor a time-dependent collapse of matter~\cite{Frolov:2015bia}. In the quantum context, the interplay between an 
%{\it exponentially enhanced} vertex operator and an {\it exponentially suppressed } graviton propagator ensures that, beyond 1-loop, the theory becomes finite~\cite{Talaganis:2014ida}\cite{Tomboulis:1997gg}\cite{Modesto}. 
With the precise form of the modified graviton propagator in hand, the requisite tachyon criteria were also established by a decomposition into partial fractions.

Having established the foundation for presenting a viable infinite derivative extension to GR in the preceding chapters, we then turned our attention to the main crux of the present work -- the avoidance of singularities -- in Chapter \ref{chap:sing}. The chapter began with a discussion on the nature of singularities and indeed, the difficulty in defining such phenomenon, before introducing the Raychaudhuri equation and analysing its import in the context of GR. Having established the focusing behaviour of null rays via the convergence condition and Hawking-Penrose theorems, we turned our attention to spacetimes that do not conform to this convergent behaviour. By reversing the inequality, we were able to examine the behaviour of null rays as they diverge or \emph{defocus}, within a geometrically flat framework. We dubbed this reversal the \emph{defocusing condition} and null rays conforming to this condition will not converge to a point in a finite time and are said to be null geodesically complete -- stretching to past infinity. This is in direct contrast to converging rays, where a photon travelling along a geodesic of this type will cease to exist in a finite time. We illustrated this defocusing behaviour with a known example of a \emph{bouncing} cosmology.

From the behaviour around Minkowski space, we were able to deduce a simpler form of IDG action, which was purely scalar in its modification, with which one could realise the desired ghost-free, defocusing behaviour. With this action in hand, it proved quite straightforward to rearrange the linearised field equations around de Sitter in precisely the same form as in the Minkowski case. From this vantage point, the defocusing conditions around de Sitter space, were also derived and found to conform to the Minkowski case at the limit $H\rightarrow 0$, as expected. 

This thesis presents a number of novel results by the author. Firstly, the calculation of non-linear field equations for the most general, infinite derivative action of gravity that is quadratic in curvature, \eqref{action},  had never been fully captured before the work that the chapter is based on, \cite{Biswas:2013cha}, was published. Previous work, such as \cite{Schmidt:1990dh},\cite{Quandt:1990gc},\cite{Decanini:2007zz} has centred on finite orders of the D'Alembertian acting on the curvature scalar.

The form of the IDG-modified propagator around Minkowski space was established in \cite{Biswas:2010zk}, along with the associated ghost-free condition. The present work reaffirms these results, while also extending them in to de Sitter space, in a novel approach, by way of a simplification of the gravitational action - a reduced action that still exhibits the required defocusing behaviour. This allowed for the extension of the recent article, \cite{Conroy:2016sac}, which detailed the defocusing conditions around Minkowski space to include defocusing conditions also around de Sitter space. Comparisons were made with finite derivative extensions of gravity, where it was found that non-locality plays an integral role in realising the desired defocusing behaviour.

The methodology used in deriving the defocusing conditions is in stark contrast to previous work on bouncing solutions in infinite derivative theories of gravity. In \cite{Biswas:2005qr}, \cite{Koshelev:2012qn},\cite{Koshelev:2013lfm}, an Ansatz was invoked as a solution to the field equations, admitting bouncing solutions, with scale factor $\propto \cosh(\frac{\sigma}{2} t)$. In the present work, we make no assumption on the nature of the scale factor a priori, except that it must conform to the requirement of accelerated expansion of the Universe  within a homogeneous framework. Having acquired the generic ghost-free defocusing conditions, we do indeed check the bouncing solution $a(t)=\cosh(\frac{\sigma}{2} t)$ for consistency and, as expected, it did display the desired behaviour.
%
%The crucial significance of the main results of this work, \eqref{defocusmink} and \eqref{defocusdS}, are that these defocusing conditions allow us an insight into the constraints that must be placed on the curvature so as to resolve the cosmological singularity problem.
\subsubsection*{Future Work}
\emph{Homogeneous Solutions}\\
%It is perhaps, the bane of the thesis writer, that ideas proliferate as the deadline looms. One could say that, the more that is written, the more there is to be written.
 Within the context of a homogeneous framework, the defocusing condition \eqref{defocusmink} could perhaps be analysed for specific restrictions on the curvature. For example, we analysed a bouncing solution in Section \ref{sec:Bouncing} and it is perhaps straightforward to generalise this analysis with a generic bouncing scale factor, using the same integral form method. Such a scale factor would result in the curvature being given by an even function, i.e. $R(t)=r_0+r_2 t^2+...$. This would have some similarities to the analysis in \cite{Conroy:2014dja} where, similarly, a generic bouncing scale factor was analysed but through the prism of the diffusion equation method \cite{Calcagni:2007ru}.

More illuminating still would be to solve the inequality for all forms of curvature that may satisfy the defocusing conditions \eqref{defocusmink} and \eqref{defocusdS} -- to see, explicitly, whether there curvature must conform to a bouncing scale factor or whether other solutions do exist. In this way, we could conceivably build up a precise form of non-singular metric, which would always satisfy the desired defocusing behaviour.  

Another quite straightforward approach in the homogenous setup would be to extend the methodology to non-linear FRW. Whereas the generic defocusing conditions can be derived without difficulty, some issues remain in terms of the ghost-free conditions. Recall that the ghost-free conditions and modified propagator were derived within a background of constant curvature. As the curvature in an FRW background is a time-dependent function rather than a constant, these conditions must be generalised to make revealing predictions. One possible method would be to proceed with an analysis that stipulates slowly varying curvature.

Furthermore, an extension of the progress made in \cite{Vachaspati:1998dy} to include bouncing cosmologies could be particularly illuminating. Vachaspati and Trodden found that the convergence condition \eqref{nullCC} restricted trajectories passing from normal regions to antitrapped regions, detailed in Fig. \ref{bigbang}. It would be interesting to see, geometrically, if a relaxation of the convergence condition allows such behaviour. One could also trace the trajectories of rays starting out at past infinity in Fig. \ref{figbounce} to shed light on the behaviour at times in and around the bounce. 
\\\\\emph{Other Solutions}\\
A further avenue of exploration involves extending our defocusing analysis to include \emph{inhomogenous} solutions, with spatial as well as temporal dependencies. This was briefly covered in \cite{Conroy:2016sac}, where the inhomogeneous generic defocusing condition was given by
\[
\frac{f(\bar\Box)}{a(\bar\Box)}\left(\partial_{t}^{2}+\partial_r^2\right)R^{(L)}<0
.\]
%Here, without loss of generality, we considered only the perturbations along the $x$-direction, where $r=\sqrt{x^2+y^2+z^2}$ and 
As before, we required $T_{\mu\nu}k^\mu k^\nu \geq 0$ so as not to violate the NEC. Note also that $\partial_r^2=\partial_i\partial^i$ is the Laplace operator. Although, the defocusing condition can be attained in quite a straightforward manner, a full analysis remains incomplete, in that the spatial dependencies must be made tractable. Similarly, we may also wish to consider anistropic spacetimes, conforming to a general metric of the type 
\[
ds^2=-dt^2+a^2(t)\sum_i e^{2\theta_i (t)}\sigma^i \sigma^i
,\]
as in \cite{Cai:2013vm}, where $t$ is cosmic time and $\sigma^i$ are linearly independent at all point in the spacetime. This is an example of an anisotropic but homogeneous metric in four dimensions, but could conceivably be generalised further to include spatial dependencies.
\\\\
This thesis presents a concrete methodology in describing a viable non-singular theory of gravity within the framework of an homogenous cosmology. Through the analysis it is clear that non-locality, arising from IDG, plays a pivotal role, as does an additional degree of freedom in the scalar propagating sector. Such a methodology can be extended into more complex pictures of IDG, such as those described above. As we extend our study, we will understand more about the relationship between the geometry of spacetime and gravity in a non-singular spacetime. We may even broaden our analysis into the study of the blackhole singularity problem, with the overall aim of, perhaps one day, presenting a definitive picture of a non-singular theory of gravity.  


