
% Thesis Abstract -----------------------------------------------------

%\begin{abstractslong}    %uncommenting this line, gives a different abstract heading
\begin{abstracts}        %this creates the heading for the abstract page
Hierarchical models and observations show that galaxy interaction and merging is of paramount importance to galaxy assembly and evolution. However, the relationship between these physical processes and the characteristics of the galaxies involved is unclear. In this thesis, we make direct constraints between the physical processes occuring in galaxy interaction - increased star formation, nuclear activation, and morphological disturbance - and the underlying parameters of galaxies - their interaction stage, stellar masses, kinematics, and orientation parameters. 

To constrain this relation, we require large samples of interacting galaxies which are representive of the full underlying parameter space. First, we aim to build this sample. A clear signature of interaction or merger activity is through the morphological distortion of a galactic system. We search for such signatures through the entire \textit{Hubble} Space Telescope (\textit{HST}) science archive. This is only possible with ESA Datalabs. In total, we classify 92 million sources into interacting and non-interacting galaxies. We find 21,926 disturbed and interacting galaxy systems; the largest interacting galaxy sample morphologically classified to date. 

We use this new sample to explore the relationship between interaction stage and numerous galactic parameters. We cross match our new sample with the Cosmic Evolutionary Survey and find ancillary data for 3,829 interacting systems. We find radical changes in the star formation rate of our sample with stage, with the complete disappearence of the red sequence at the merging stage. We find that the fraction of galaxies with an active galactic nuclei is constant with interaction stage, except at the point of coalescence where we find it dramatically increases. By investigating the relationship between these fundamental processes and stage, we show that there is a direct relation between the dynamical timescale and these processes.

Thus, we introduce a new algorithm which will be capable of exploring this relationship. We utilise a three-body numerical simulation with a Markov-Chain Monte Carlo (MCMC) algorithm to directly map the parameter space to morphological disturbances. We constrain the underlying parameters of a sample of 51 synthetic images representing different observed interacting systems. We are able to recover the true parameters within a confidence interval of 86.4\%. We apply this methodology to a subset of observational data, and explore this algorithms potential as well as its limitations.

\end{abstracts}

%\end{abstractlongs}


% ---------------------------------------------------------------------- 
