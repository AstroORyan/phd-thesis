\chapter{Introduction}\label{chapter:introduction}
% Introduce Galaxy Interaction, why is it important?
In the context of large scale structure of the universe, galaxies represent a fundamental building block of matter. Their evolution through cosmic time \citep{2005Natur.435..629S} leads to the local universe as we see it today. A key part of that evolution is that of mutual interactions between galaxies. Our best cosmological model, $\Lambda$-Cold Dark Matter, dictates that galaxies assembled hierarchically \citep{1978MNRAS.183..341W, 1991ApJ...379...52W}. Therefore, we must understand the effects of mutual interaction to not just understand galaxy evolution but to better understand our theories of cosmology.

We have come to understand that interaction has multiple impacts on the evolution of galaxies. The first, and most obvious effect, is that it leads to the distortion of the disks involved and the formation of distinct tidal features. Since initial simulations of close pairs of galaxies, these morphological features have been re-created with excellent accuracy \citep{1972ApJ...178..623T}. Often, we attempt to identify interacting galaxies from their change of morphology. However, if they exhibit little distortion we require redshift information to ascertain their 3D distances from each other. However, spectroscopic information across the many million close galaxy pairs we know about is limited.

From the systems we have reliably identified, we have found mutual interaction has further effects on galaxies than simply morphological disturbance. It is widely debated about the role of interaction in nuclear ignition and enhancement in star formation of each system. While the former may not happen at all, the latter is generally accepted to occur while the specific strength and impact on the system remains highly debated. Such a change in the star formation therefore leads to significant changes in the gas reserves in a galaxy, leading to the potential quenching of the system post-interaction. However, in the inverse of this, interaction between gas-poor and gas-rich systems has been shown to rejuvenate the galaxies involved, and bring them out of a quenched state and into a star forming state. The scale, change and impact of these effects is dependent on a host of underlying parameters and specifics of the interaction in question. However, the link between the fundamental parameters of interaction and the characteristics of the final system have been poorly explored. In this work, we will explore this relationship in the context of using large-scale samples and with directly comparing to simulations. However, for now, it would be prudent to begin this discussion by exploring the definition of  a galaxy, and outlining the major work already conducted into linking galaxy interaction to its underlying parameters and, furthermore, with galaxy evolution.

\section{WHAT IS A GALAXY?}
\noindent For the purposes of this work, a galaxy will be the smallest unit of mass we will consider. A galaxy is an extended system of stars, gas and dust which all orbit around a supermassive black hole (SMBH) at its centre. All of this baryonic matter makes up the luminous, or visible, matter of the galaxy. This matter has two distinct components: the galactic bulge and the galactic disk. The galactic bulge lies at the centre of the system. It is an approximately spherical component which extends, typically, out to about a third of the radius of the galaxy. Galaxies will always have a bulge component, although it might be so small that the galaxy appears to only have a disk. The bulge is often made of an older population of stars, with high peculiar motions in their orbit. At the centre of the bulge, we find the SMBH. Often, this SMBH is quiescent. However, under certain circumstances the black hole can start accreting lots of gas. Whether driven by galaxy mergers \citep{2006ApJS..163....1H, 2014MNRAS.438.1839B} or secular processes \citep{2012ApJ...757..179A, 2017MNRAS.470.1559S} the result of this black hole feeding can cause an ignition of so-called `nuclear activity' and its growth. When the SMBH is in this state it produces large energetic jets of material and radiation perpendicular to the galaxy. When galaxy's are observed with this active SMBH, it is classified as an active galactic nuclei (AGN). About the the SMBH and galactic bulge lies the galactic disk. The galactic disk is a very thin, extended region orbiting about these central components. This is where the majority of the galaxy's stars, gas and dust reside. There are many younger stars throughout it as well, and this extends out to a radii many times that of the galactic bulge. Both of these are embedded in a dark matter halo.

The dark matter halo is many times the radius of the luminous components of the galaxy. This halo is also many times the mass of baryonic components of the galaxy, and makes up the majority of the mass of a galaxy. What this dark matter is, specifically, is still a very well researched topic with many different candidates which we will not explore here. Each these component is modelled with different potentials. For the galactic bulge, we utilise a Plummer sphere \citep{1911MNRAS..71..460P}. For the galactic disk, we use an exponential disk model. For the dark matter halo, we utilise a cored Navarro-Frenk-White model \citep{1997ApJ...490..493N}.

For much of the history of extragalactic astrophysics, the idea that galaxies could interact and merge was thought to be very inprobable outwith dense clusters. The small radii of the luminous matter in galaxies, and their relatively small calculated cross sections, led astronomers to believe that the probability of an interaction was close to 0 and that galaxies were, simply, island universes  \citep{1926ApJ....64..321H}. However, with the development of the idea of the dark matter halo, with its radius many times that of the luminous matter, it was realised that the probability of galactic systems interacting and merging was within the realm of possibility. And, therefore, over time the idea that galaxy interaction and merging would have a significant impact on galaxy evolution was developed. In fact, the merger rates of galaxies through cosmic time is a well studied concept of our cosmological models with observations and simulations.

\section{GALAXY ASSEMBLY ACROSS COSMOLOGICAL TIME}
\noindent Galaxies, throughout cosmic history have been assembling and accreting matter to form the universe that we see today. Studies at high redshifts reveal that galaxy structure is significantly different compared to the local volume. Initially, galaxies were very small systems that formed from gravitational instability thoughout the early cosmos \citep{1993MNRAS.262..627L}. These instabilities were generated by the physical processes of the very early universe brought about by the small scale of the universe. From these initial density perturbations arose small systems which accreted from the gas and dust. These early galactic systems were much smaller than those we see today. Their morphology was also very peculiar, and they were forming stars at a much higher rate than the present day. While these systems were accreting much of their matter to gain mass, the universe was expanding. At this time the rate of galaxies interacting and merging was also greater than today \citep{2010ApJ...715..202H, 2011ApJ...742..103L}. This provided a peak in the cosmic star formation of the history through 1 $<$ z $<$ 3 where approximately half of all stellar mass was formed \citep{2005ApJ...625..621B}.

Figure \ref{fig:merger-rate} shows the changing merger rate through cosmic time. As the merger rate increased, peaked and then declined it had profound effects on both the star formation density of the universe and the morphology of those galaxies within it. From z = 1, the once massive galaxies, with rapid star formation, have transformed into bulge-dominated galaxies contraining super massive black holes \citep{2007ApJ...654..858B}. In fact, as we go to lower redshifts, we find that the majority of the stellar mass is contained within these systems. These galaxies are dominated by old stellar components and have little ongoing star formation \citep{2002AJ....124..646H, 2004ApJ...608..752B}. Thus, showing that mergers were the primary drivers of star formation in the past. 

\begin{figure}
    \centering
    \includegraphics[width=0.95\textwidth]{Introduction/figures/merger-rates.png}
    \caption{The clearly declining merger rates from z = 3 to z = 0. This is Figure 3 of \citet{2010ApJ...715..202H}. This study looked at the history of a simulated set of galaxies of various stellar mass, and investigated the decrease in the merger fraction as a function of redshift and baryonic mass. As shown, for all mass bins and methods of identifying mergers (the different lines) the merger rate decreases. This is only not true for mergers between very high mass systems and low mass systems (in the bottom right of this plot). Thus, mergers between high and low mass systems may still have some driving force in the cosmic star formation rate density.}
    \label{fig:merger-rate}
\end{figure}

This gives us an avenue by which to link an observed galaxy's morphology to its merger history, and split them into two distinct populations. One population is of galaxies with a history of intense interactions with galaxies of equivilent mass, which lead to the complete destruction of their disks, and the other is of galaxies with a less intense merger history with systems of much smaller mass than their own. These interactions and mergers serve to feed the gas within the galactic disks and increase their mass, while preserving that component of the system. Thus, starting from the observations of morphology, we can make assumptions about the internal gas content, star formation and, therefore, colour of galaxies. In the next section, we will describe how these links are made and the fundamental properties that make this link.

\section{GALAXY MORPHOLOGY \& GALACTIC PROPERTIES}
\noindent Galaxy morphology, and its change with time, specifically is the study and measure of a galaxys' shape and features. A striking distinction in galaxy morphology is been between disk and elliptical galaxies, the primary breakdown in the famous `Hubble tuning fork' shown in Figure \ref{fig:hubble-tuning} \citep{1936rene.book.....H}. Once, this was thought to be a map of galaxy evolution with early-type elliptical galaxies on the left which would evolve to the late-type disk and spiral galaxies on the right. However, this was found not to be the case with early-type elliptical galaxies being older than late-type disk galaxies. So, while this classification scheme was found to not be a direct evolutionary pathway, each classification was found to be indicative of the merging history of a galaxy. Elliptical galaxies are bulge dominated ones, with high internal velocity dispersions and a spheroidal shape. Such systems are often created as the result of mergers and cannibalism of smaller systems to interactions with counterparts of a similar mass \citep{1996MNRAS.283.1361B, 2006MNRAS.366..499D}. Disk galaxies, on the other hand, are rotationally-dominated systems with a central bulge whose size is crucially dependent on the merger history \citep{1992ApJ...393..484B, 2010ApJ...715..202H, 2017ApJ...837L...8B}. In fact, galaxies that appear to have no bulge component at all have had no merger event in the last few Gyrs \citep{2012ApJ...756...26M}.

\begin{figure}
    \centering
    \includegraphics[width=0.95\textwidth]{Introduction/figures/hubble-tuning-fork.jpg}
    \caption{The Hubble tuning fork. Credit goes to the Galaxy Zoo collaboration (and specifically Karen Masters) for the creation of this figure. What was once thought of as an evolutionary track for galaxies, is now a widely adopted classification system for them based on morphology. On the left, we have elliptical systems - so called early-type galaxies - which are often `red and dead' systems. They have an intense merger history, which has destroyed their disk component and caused them to use the majority of their gas in star formation. On the right, we have two different kinds of disk galaxies. Ignoring the bar or un-barred part of this, they are systems which have a less intense merger history and have only accreted smaller systems into them. This serves to enhance their gas disk, and preserve their disk component.}
    \label{fig:hubble-tuning}
\end{figure}

This is also reflected in the observed colour of disk and elliptical galaxies. Observational colours in this context are the comparison of flux in two different wavelength ranges: one capturing the flux of young, star forming regions while the other captures the flux from old, low mass stars. In combination, these young and old stars, constitute a stellar population. Stellar populations have well defined spectral energy distributions (SEDs) based on their age and, indirectly, to a galaxys' star formation rate (SFR) and gas mass. When star formation is occurring rapidly, and the SFR is very high, lots of young, luminous OB-type stars form. The flux from these stars falls mainly in our blue filters, in the wavelength range up to the ultraviolet. These stars have lifetimes of only a few million years and, therefore, die very quickly. Thus, if stars are forming slowly, and the SFR is very low, these OB-type stars will die and not be replaced. This leaves an older stellar population, mainly composed of M-type stars which lie primarily in our red filters, with wavelengths in the optical. Thus, the SFR of a galaxy very much influences the colour of galaxies.

The SFR, then, is highly dependent on the gas mass present in the galaxy. This gas must be molecular gas, with little energy so it is able to form massive clouds of self-gravity and undergo collapse \citep{1965MNRAS.130...97G, 1972ApJ...176L...9Q}. From models of individual clouds, it was noted that the gas density was related to the density of star formation in galaxies \citep{1959ApJ...129..243S}. This was further refined to be a global law in disk galaxys' in particular in \citet{1998ApJ...498..541K}. In this, it was shown that the surface area of star formation is directly related to the surface area of gas by 

\begin{equation}\label{eq:ks-law}
	\Sigma_{SFR} \propto \Sigma_{Gas}^{n}. 
\end{equation}

\noindent This relation has been found to be n$\approx$1.3. Thus, two things are happening here, if a galaxy has gas, it has star formation and if it has star formation, it will be observed to be blue and vice versa.

It has been found that if a galaxy is elliptical, it is likely that it is much redder than a disk galaxy \citep{1992MNRAS.254..589B}. These galaxies have a more violent merger history that has destroyed the ordered rotationally dominated component of a galactic disk and, in the process, either removed or used up the gas within the galaxy \citep{1976ApJ...204..365F}. As a result of this, the current gas mass and density are very low which leads to the SFR being very low. Thus, the stellar populations that make up elliptical galaxies are much older when compared to disk galaxies.

% Talk about the blue-ness of disk galaxies.
The opposite is true for disk galaxies. As stated previously, these systems have had a less tumultous merger history - particularily since z $\approx$ 1. An indicator of the merger or interaction history in disk galaxies is, in fact, the size and shape of the bulge component \citep{2011MNRAS.414..888E}. However, because of the plentiful gas, dust and ordered rotation in the galactic disk the SFR within is larger than elliptical galaxies. The surface density of gas within the disk is much larger than in elliptical galaxies providing the ideal for star formation. Spiral arms can exist in such galaxies, containing large filaments of gas and dust. They contain areas where large molecular clouds can condence, fragment and then collapse into new stars. These star forming regions will be young enough that massive OB-type stars will form, which in turn changes the underlying SED of the galaxy that we observe. The blue filter will contain more flux, and therefore, the galaxy appears blue when comparing the red and blue filters.

\begin{figure}
    \centering
    \includegraphics[width=0.95\textwidth]{Introduction/figures/population-distribution.png}
    \caption{The distribution of galaxies in colour-colour space showing two distinct populations in a clear bi-modal structure. This is using the u-band and r-band filters and, therefore, the closer to 0 on the y-axis the bluer the galaxy. Left plot: the top population is the blue cloud, primarily composed of disk galaxies with young, star forming stellar populations. The bottom population is the red sequence, primarily composed of more massive elliptical galaxies which are gas poor and quiescent. Between these two populations lies the green valley, marked by the green lines. This is believed to be a transitional population moving between the blue cloud and red sequence for debatable reasons. To further show the split with morphology, the panels on the right split the colour-mass space into early-type (elliptical) galaxies and late-type (disk) galaxies. Note, this is Figure 2 of \citet{2014MNRAS.440..889S}}
    \label{fig:blue-red-population}
\end{figure}

% Now, talk about the blue and red sequence.
This gives rise to two distinct populations of galaxies: old, red elliptical galaxies and young, blue disk galaxies. When plotting colour-colour or colour-magnitude diagrams, there is clear bi-modality which separates these two populations with the `green valley' running between them \citep{2001AJ....122.1861S}. Figure \ref{fig:blue-red-population} shows the colour-colour distribution if the two populations. In this Figure, we clearly can see the blue population - blue cloud - and the red population - red sequance - beneath it with the green valley in between. The green valley is a disputed area of this distribution. Some works claim that it is a transition phase \citep{2007ApJS..173..315S, 2015MNRAS.450..435S}, where galaxies are moving between the two populations due to different processes over a short period of time. Others, meanwhile, dispute this and that star formation is either very quickly cut off or declines very slowly \citep{2014MNRAS.440..889S}, with the green valley being a by-product of the latter evolutionary pathway. However, what is definitive is that the rule of red galaxies are elliptical and blue galaxies are disk galaxies is not ubiquitous \citep{2022MNRAS.510.4126S}. Examples of blue ellipticals and red spirals do exist \citep{2009MNRAS.396..818S, 2010MNRAS.405..783M, 2022AJ....163..150K} and they are the subject of intense debate and study in the current field. However, as a general guideline for the expected properties of a galaxy the colour and morphology are deeply interdependent on one another.

% Want to introduce the idea of starburst galaxies, remnants and such here.
There are also many types of systems that do not fit this simple morphological definition of galaxies. Galaxies defined as starburst galaxies are often very morphologically irregular and have measured SFRs far in excess of expectation for their mass. These then lead into post-starburst galaxies and merger remnants, who are the complete opposite and have measured very low SFRs. These are often called quiescent galaxies. These highly irregular systems have direct interplay, however, with their merger histories and surrounding environments \citep{2018MNRAS.477.1708P, 2020MNRAS.493.3716H}. In fact, it has been found that in cluster environments, the fraction of disk galaxies drops to $\leq$10\% when compared to $\approx$60\% in the field.

The effect of the galactic environment on a galaxy should also not be understated. There are many ways to define the galactic environment, but it is most commonly associated with the number density of galaxies about them \citep{2003ApJ...585..694E, 2004ApJ...615L.101B}. There are three broad classifications of environment: field, filament and cluster. A field galaxy is one with neighbours, and is in relative isolation. A filament galaxy has some neighbours, but their influence on the galaxy in question is minimal. A cluster galaxy is one in which a large number of other galaxies are bound in colossal galaxy cluster structures. In these environments, SFRs are suppressed \citep{2006MNRAS.373..469B} by ram pressure stripping and galaxy morphology is highly irregular due to constant interaction and merging. In these environments, the relation between galaxy morphology and the underlying processes is almost completely broken by interference from the environment.

Thus, galaxies in the field or in filaments are of particular interest. Here, we can study the direct link between morphology and the underlying processes of galaxies. And these, in turn, are highly dependent on galaxies underlying merger and interaction histories. So, what is this relation? As stated previously, the influence of interaction and merging accelerates the evolution of rotational systems to dispersion dominated systems. However, there are many more subtle affects that can be attributed to galactic merging and interaction. For instance, if two galaxies can be brought into a state of starbursting due to merging \citep{2008MNRAS.385L..38M}. This completely changes the underlying SED of the galaxies involved, and can lead to the rapid quenching both systems \citep{2018MNRAS.476.2591V, 2022MNRAS.517L..92E}. Thus, depending on when we observe such a galaxy, we can find either a much bluer or redder galaxy than expected \citep{2007A&A...468...61D}. However, these effects are also debated, with some studies finding that galaxy interaction does not induce significant change in the SFR, and therefore, the underlying SED of the galaxy population \citep{2003A&A...405...31B}. So, why is this? It transpires that the effects of a galaxy merger or galaxy interaction are dependent on the underlying parameters of the galaxies themselves. This leads to us having multiple classifications of galaxy interactions which each lead to different outcomes for the galaxies involved. 

\section{CATEGORISATION OF MERGERS}
\noindent The effects of galaxy interaction and merging is dependent on multiple different factors and underlying parameters. Key parameter to the destruction or survival of a disk is the mass ratio between the two systems \citep{2008MNRAS.384..386C}. The impact parameter also does have a lot of influence over the final morphology over the system, however, we will discuss this further in Section \ref{sec:int_effects}. The gas content of each galaxy also has an effect on the changes of the internal SFR of the systems. The large change in the systems we see based on these parameters gives rise to many different categorisations of galaxies. For the mass ratio, we define a major, a minor and a micro interaction. These are the ratios of the primary galaxy mass (the more massive galaxy) and the secondary galaxy mass (less massive galaxy). These are then further sub-divided into two seperate categories based on their gas content: a wet merger or a dry merger. This refers to the gas mass and colour of both galaxies. A wet merger involves lots of gas in the two galaxies colliding, and therefore, lots of resultant star formation. Both of the involved galaxies are blue, and often disk-dominated galaxies. A dry merger involves little gas, and is often the merger of massive, red elliptical galaxies. There are also intermediate categorisations based on the amount of gas, such as damp mergers (where there is some gas, but not enough to completely dramatically increase star formation) and mixed mergers (mergers between gas-poor and gas rich systems), however, we focus on the binary definition of wet and dry mergers.

Depending on the classification of an interaction leads to specific outcomes for the interacting galaxy system. A major interaction is when the galaxies involved have a mass ratio of approximately 1:1. Some definitions vary, however, and allow this limit to go down to 1:3. For the purposes of this work, they must have a mass ratio of approximately 1:1. Major interactions are the most devastating to the morphology of the systems involved, with the forces upon them causing severe morphological distortion and the formation of tidal features (more on this in the following section). If the two systems merge, there is complete destruction of the disks in both galaxies and the post-merger remnant will be highly irregular. If this major interaction is also a wet one we would also expect a significant increase in the SFRs of both galaxies - this is particularily true if the two systems actually coalesce \citep{1994ApJ...425L..13M, 1996ApJ...464..641M, 2006AJ....132..197W}. We also find that wet major mergers have an increased AGN fraction, which suggests that such a merger may play a role in nuclear ignition. The opposite of this is true when a major interaction is dry, where only a small increase in SFR is found in the nuclear region of the galaxies \citep{2006ApJ...640..241B}. This holds true even when the two systems actually merge, with no significant change in SFR or AGN fraction emerging at all.

\begin{figure}
    \centering
    \includegraphics[width=0.95\textwidth]{Introduction/figures/combined-examples-mergers.jpg}
    \caption{Examples of a major, a minor and a micro interaction. Each of these are wet interactions, containing lots of gas and therefore increases in star formation. These are the Arp 240, Arp1 188 and NGC 5907 systems, respectively. Here, we show the famous double looped stellar stream of NGC 5907 but point out that the existence of the second loop is in dispute and that we only show it here for illustrative purposes. From major to micro interactions, we see decreasing impact and change in the morphology of the primary galaxies but always the complete destruction of the secondary.}
    \label{fig:merger-clsfs}
\end{figure}

A minor interaction has less catastrophic consequences for the morphology of one of the two galaxies. This is defined as an interaction where the mass ratio is less than 1:1 but greater than 1:3. Due to this large mass disparity, the morphology of the primary galaxy is relatively unaffected by such an interaction. However, the secondary galaxy will be almost completely destroyed by the encounter. If it is only flying by in an interaction, the secondary will be higly disrupted, forming a long and stretched tidal tail as it moves through the orbit. When these systems actually merge, we observe small increases in the SFR of the primary. There is ongoing work investigating whether such mergers were actually a primary driver of star formation in galaxies across cosmic time through rejuvenation of gas resevoirs within the primary galaxy \citep{2007A&A...476.1179B, 2014MNRAS.440.2944K, 2022MNRAS.511..607J}.

Finally, a minor interaction is one in which the mass ratio between the primary and secondary galaxies is less than 1:3. This always sees the complete destruction of the secondary galaxy, whether a flyby or a complete merger. These also form many stellar streams about their primary galaxy and are often absorbed by the primary with very little change in its morphology. Therefore, it has been theorised that such mergers could be a much larger driver of cosmic star formation through time but leaving very little evidence or impact on the primary galaxy. 

Figure \ref{fig:merger-clsfs} show examples of each of the different merger categories we have defined here. These, from left to right, are the Arp 240, Arp 188 and NGC 5907 systems. They each show a major, minor and micro interaction, respectively, and demonstrate the change in effect the mass ratio has on the morphology of the galaxies involved. From Arp 240, with the complete destruction of the galactic disks and formation of tidal features to NGC 5907 where the primary galaxy is barely disturbed at all. Thus, the different categories of interactions and mergers have very different impacts on the systems involved. We will now discuss, in depth, the effects that interaction has on these galaxies and specifically explore the formation of morphological disturbances like tidal features, changes in the SFR and the increase in AGN fraction.

\section{EFFECTS OF GALAXY INTERACTION}\label{sec:int_effects}
% Here, present examples of interacting galaxies and the context in which we're talking about them.
The idea of two or more galaxies interacting and the effect that it would have had upon galaxy formation and evolution was thought to be minute. We now understand that galaxy interaction plays an important, and fundamental role in the evolution of galaxies. As stated previously, they have multiple affects upon the systems undergoing the interaction. The specific effects of interaction lies on a host of parameters, we have discussed the mass ratio and gas content and mentioned the impact parameter but there is also the orientation of the interaction, the relative sizes of the galaxies and the point in the dynamical history of the interaction we are observing. Thus, in this section, we will explore how these different underlying parameters link to the physical processes we observe in interacting and merging galaxies.

% For each of these sections below, I want to do a deep dive on how each process is driven. Will maybe mention simulations throughout, but we shall see?
\subsection{Morphological Distortion: Tidal Features}
\noindent It is important to start here with pointing out that galaxies are large gravitationally bound systems of stars, interstellar gas, dust and dark matter. Upon a close encounter between two galactic systems, these components experience strong gravitational forces which disrupt their ordered layout. As the galaxies move through their relative orbits, the gravitational potential changes at an accelerating rate. This imparts energy into the ordered components of the galaxies, and causes thermalization of their internal motions. This leads to violent relaxation, where the stellar orbits are so altered they no longer follow their prior orbits. However, the energy in the system must be conserved. Thus, as the internal stellar motions within the galaxy thermalizes, the energy in the galaxy's orbits decay via dynamical friction. If this decay is large enough, the energy in the galaxy's orbit may be sufficiently reduced to no longer escape from one another and they will eventually coalesce to leave a single merger remnant. 

However, before this, the changing graviational fields as the two galaxies encounter each other leads to radial distortion of each galaxy. This leads to the drawing out of galactic material into long plumes which can eventually become tails. If stellar material, including gas, dust and stars, is close to the edge of the galaxy a combination of the galactic rotation and radial elongation lead to it being sheared off and away from the galaxy. This forms two `tidal tails` in the system, one leading and one proceding the motion of the galaxy. Dependent on the geometry of the encounter, the trailing tail of one galaxy and the preceding tail of the other can form a `tidal bridge' - a linking of the two systems. An example of both of these is the interacting pair Arp 240, shown in left hand plot of Figure \ref{fig:merger-clsfs}. 

The geometry of the bulk motion of the galaxy is very important in forming these tidal features. As the internal galaxy rotation and bulk motion in of the galaxy's orbit in the encounter must match for this shearing of material to occur. Thus, the formation of tidal tails and tidal bridges is only possible in a prograde interaction whereas a retrograde interaction suppresses them. Further, depending on the relative velocities in of the galaxies these tidal tails can be split off from the galaxy as the encounter continues. This can lead to the formation of `tidal debris' of stellar material forming about the galaxies.

The confirmation that features such as these were from a tidal origin came primarily from simulations of interactions of different systems. The first restricted numerical simulations were conducted by \citet{1972ApJ...178..623T} who successfully modelled the features of four different systems using distributions of test particles. Since this, numerous other works have recreated tidal features using numerical simulations \citep{1993ApJ...410..586S, 2008AN....329.1046P, 2009AJ....137.3071B, 2016A&C....16...26W}. These have also been expanded to include a range of other properties in hydrodynamic simulations \citep{2021MNRAS.503.3113M}, and investigated in cosmological simulations \citep{2020MNRAS.493.3716H}. However, even in these more realistic simulations, the key to linking the simulations to observations is the morphology of the tidal features. Often, this is from either direct comparison or from searching for analogues through a suite of interaction simulations in cosmological simulations.

While the most striking tidal features to form in galactic encounters are these tidal tails and bridges, we also see the formation of a slew of other features. These include `stellar streams' (previously discussed), shells and rings forming in the galactic disk. As stated, stellar streams are likely smaller galactic systems that have been destroyed while passing the primary galaxy. This leaves a faint stream of material about the galaxy. The right plot of Figure \ref{fig:merger-clsfs} shows an example of a stellar stream with the NGC 5907 system. Shells, on the other hand, are formed about galaxies and can be present in as many as 10\% - 20\% of elliptical and lenticular galaxies \citep{1983ApJ...274..534M, 2013ApJ...765...28A}. The left panel of Figure \ref{fig:tidal-features-ex} shows an example of shell in the elliptical galaxy NGC 1344. Spectroscopy shows that shells are primarily composed of stars. There are often numerous shells in a single galaxy, although most often we only detect $\approx$3. Shells are formed from the disruption of a small satellite around a significantly more massive galaxy - so, in a minor to micro interaction. This then forms a stellar stream, which overtime condenses to a cloud of stars orbiting the primary galaxy. This then either forms into an X-shaped structure or a annulus which, in the 2D projection of the sky, appears as a shell. Thus, observing a stellar stream or a shell is highly dependent on the time in the dynamical history of the encounter that we are observing.

\begin{figure}
    \centering
    \includegraphics[width=0.95\textwidth]{Introduction/figures/shells-rings.jpg}
    \caption{Examples of a collisional ring galaxy and a system containing shells. Left: The Cartwheel galaxy, the most famous ring galaxy to date. Ring galaxies can only formed by a direct, head on collision between two galaxies. The ring itself is composed of a region of the disk undergoing intense star formation due to the density wave passing through the disk from the impact of the secondary galaxy. Right: The system NGC 1344, showing two shells of stellar material about it. This is formed by the condensation of stellar streams into clouds of stars. Cartwheel Galaxy Image: NASA / Hubble, NGC 1344 Image: \citet{1983ApJ...274..534M}}
    \label{fig:tidal-features-ex}
\end{figure}

Finally, we find rings can form from galaxy interaction. These systems are often called ring galaxies or collisional ring galaxies. Figure \ref{fig:tidal-features-ex}'s right panel shows the Cartwheel Galaxy: a famous example of a ring galaxy. This ring is formed of young, hot stars and is formed only from the head-on collision with another system \citep{1976ApJ...209..382L}. The interaction to form this is so intense, that it causes a density wave to pass through the galactic disk which triggers intense star formation at its wake. These systems are incredibly rare, as this requires an impact parameter of 0 for this feature to form.

From the example of a ring galaxy, we see that interaction not only affects the stellar distribution of the galaxy, but also has direct impacts on the gas within it. As we have stated in previous sections, interaction and merging can induce enhancements in star formation and even lead to a starburst in a galaxy which brings about the complete quenching of the system.

\subsection{Star Formation Enhancement} 
\noindent As stated previously, it is often observed in interacting and merging galaxies that the star formation is enhanced in some way. While to what level this enhancement is often debated between observations \citep{2003ApJ...582..668B, 2008MNRAS.385.1903L, 2011MNRAS.412..591P, 2022ApJS..261...34H} and simulations \citep{2007A&A...468...61D, 2008MNRAS.384..386C, 2013MNRAS.430.1901H, 2021MNRAS.503.3113M}, the underlying processes that lead to enhancement is well understood. First, by star formation enhancement we mean that the SFR either globally or in different regions of an interacting galaxy is higher when compared to isolated galaxies. Thus, some property of interaction is causing an increase in the star forming activity of the galaxies involved. 

The rate at which stars form is known to be directly related to the surface density of gas at any given piont within the galaxy, sd shown in equation \ref{eq:ks-law}. It is therefore easy to see that due to an interaction the torques and gravitational forces upon the gas clouds within each galaxy causes them to lose angular momentum. This loss in angular momentum causes the gas clouds to drift inwards, towards the galactic core. As more gas clouds fall into the core, the surface density of gas in this region is driven up rapidly. This in turn drives up the SFR dramatically.

While this is a simple explanation, and an easy idea to hold about driving the increase in star formation in interacting galaxies. So, what drives these? In fact, it is the same basic principle, except due to the distortion of the disk itself. As the morphology of the disk is compressed into a tidal feature, we see the same increase in the surface density of gas which leads to a further dramatic increase in the SFR. 

This movement and compression of gas into the galactic centre also has a serious significant effect. It can lead to the feeding of the supermassive blackhole at the galactic centre. This, then, causes ignition of the supermassive black hole and us to observe an AGN in the galaxy.

\subsection{Ignition of Active Galactic Nuclei}
\noindent The presence of an AGN in a galaxy is often very hard to miss. When we view them face on, the AGN is so luminous that it often outshines the entire bulge and disk of the galaxy. Thus, to study the galaxies they are present in we must carefully remove all contribution of the AGN to the underlying SED flux. The bright flux we observe is from a long and complicated process of accretion of material onto the supermassive black hole at the galactic centre \citep[for an excellent breakdown of the structure and evolution of AGN see][]{2012agn..book.....B}. The structure of material about the black hole is also complex, and results in many observational oddities. 

An accretion disk of material that is directly rottating about and orbiting the black hole structure exists. As the material accretes, it causes the SMBH to project large, powerful jets of radiation perpendicular to the accretion disk plane. These jets are highly luminous, and are what we observe in the optical, infrared and X-ray. There has been discovered two populations of AGN: containing narrow and broad emission lines, named Seyfert 1 and Seyfert 2 AGN respectively. While, at first, thought to be two distinct types of AGN, there has now been a significant push to unify these AGN into one population. This unified population is related to the structure of material surrounding the accretion disk and the inclination of the galaxy \citep[for a review of the unification, see][]{2015ARA&A..53..365N}.

About the accretion disk, there is a torus of material. This material is dusty, and highly absorbing of the emission from the SMBH-accretion disk system. This leads us to only observing narrow line emission (Seyfert 2) when we observe the AGN through the torus and observing both narrow and broad lines of emission not viewed through the torus. As we move to higher luminosities, we get further populations of AGN such as blazars, quasars, radio-quiet and radio-loud systems. These are all dependent upon the inclination at which we are viewing the AGN system and the material in the line of sight.

Thus, from this description of an AGN structure it is perhaps easy to see how galaxy interaction increases the probability of nuclear ignition. AGN require large volumes of gas and dust to be present at the galactic core; a surge of which interaction provides \citep[][provides an excellent summary of this process from the point of view of simulations]{2008ApJS..175..356H}. However, what is often surprising from observations of samples of interacting systems is the meagre increase in the AGN fraction within them. There have been many hypothesis as to why this might be, from a delay in the AGN ignition \citep{2011MNRAS.418.2043E} to AGN flickering \citep{2015MNRAS.451.2517S}. A delay in AGN ignition would make sense, again, in the context of the structure we have just discussed. As gas and dust are moved into the galactic core, they do not necessarily immediately move dierctly around the SMBH and may take time to get there. There may be other mechanisms at work within the AGN itself that causes suppression of igntion for some time. This could also produce the AGN flickering, where material may be cutoff from the SMBH and cause the AGN to turn on and off. This could be further proved if it was found that the AGN fraction varied across the dynamical time of interaction.

It cannot be disputed that the sudden movement of gas into the galactic centre during interaction does, indeed, seem like an obvious way for activation to occur. However, as gas is removed from the galactic disk from either ejection or star formation, we see the beginning of quenching in the galactic system. 

\subsection{Quenching of Galactic Systems}
\noindent Through a galaxy interaction, as discussed, there is significant movement in gas, AGN activation and large increases in star formation across the galaxy. However, this comes at a cost. A galaxy only has a limited resevoir of gas available to it. Therefore, this sudden increase in gas usage in multiple different processes leads to the gas being used in a significantly shorter timescale than expected. Once the gas is used, star formation and AGN will rapidly cease and the colour of the galaxy will change from highly blue to highly red. This process is called quenching.

This rapid quenching of systems is different to what is expected in the field. Galaxies gradually use their gas over long periods of time forming stars and slowly quench as the average gas density across the disk reduces to the point of very low SFRs \citep{2010ApJ...721..193P}, so-called `mass quenching'. A system can also be quenched by the environment in `environment quenching'. However, by concentrating gas and increasing the gas density in an interaction, even if a galaxy had stopped forming stars previously it is able to increase in star formation again and use the remaining gas entirely. Thus, we often seen increases in the SFR during and immediately after an interaction but then the sudden shut off of SFR shortly after it \citep{2022MNRAS.517L..92E}. The sudden use of gas is not just from it being used in star formation, but also from stellar feedback around the areas where the stars are forming. As these new stellar populations form OB-type stars very quickly die and cause an increase in the rate of supernova. This, in turn, creates strong shocks and winds that drive the molecular gas in the star forming regions out of the galaxy at very high speed \citep{2013Natur.499..450B, 2018ApJ...864L...1G}. A similar process can occur due to nuclear activation from interaction. AGN are incredibly powerful, and drive strong winds across the galactic core. These winds can, similary, drive out \citep{2014A&A...562A..21C,  	2016Natur.533..504C, 2018MNRAS.480.3993B} and cause the cessation of star formation.

\section{IDENTIFYING INTERACTING AND MERGING GALAXIES}
\noindent We have described and explored the effects of interaction and merging across the galaxy population. We have described the basis of galaxies, and how these are then changed and what we expect to see when we observe interacting galaxies in comparison. However, we are only able to so fully understand these differences by using large samples of interacting galaxies and comparing them to large control samples.

% This section is going to be re-written as I talk at length about these catalogues in the next section.
The first samples of interacting galaxies were identified by eye. The earliest interacting galaxy catalogue was the \citet{1966ApJS...14....1A} of Atlas of Peculiar Galaxies. This contained 166 interacting systems. This was quickly added to with a 2-part edition of the Vorontosov-Velaminov catalogue \citep{1977A&AS...28....1V}, providing a further 268 systems to the original \citet{1966ApJS...14....1A} catalogue. Both of these catalogues contained only the most major interactions, where clear tidal features easily put them into the interacting galaxy classification. The tidal features of these systems were used to classify them by eye, meaning they were only capturing a specific part of the dynamical timescale. These samples were also not large enough to make statistical, representative statements about the interacting galaxy population. Thus, larger catalogues had to be created.

Catalogues made later than the previous examples also often used visual classification to find interacting and merging galaxies. The previously mentioned Galaxy Zoo collaboration created a catalogue of 3,003 interacting galaxies which were identified by citizen scientists based on morphology \citep{2010MNRAS.401.1043D}. However, a limitation was found when visually classifying galaxies at different parts of the interaction timescale. With no information on the redshift, and therefore 3D distribution, of galaxies within images it was very difficult to distinguish `close pairs' from truly interacting pairs. A close pair, in this context, is two galaxies which appear to be very close together in the sky in the 2D projection of the sky but are actually at different redshifts. This means they are not actually close together in 3D space and cannot possibly be interacting. Many interacting galaxies created by visual classification have to throw away masses of false positives from their sample \citep{2020MNRAS.492.2075B, 2022A&A...661A..52P}. As machine learning algorithms began to be used in galaxy classification, this problem spilled over.

Neural networks have began to play an increasingly central role in galaxy classification. As the size of observational samples have increased explonentially, so has the apparence of the impossibility of visually classifying each galaxy individually. Rather, a neural network can now be trained to make morphology classifications of galaxies using a much smaller subset of visually classified galaxies. A neural network is a layered structure of interconnected nodes which can pass information to one another. Each node has a weight assigned to it which allows it to alter the input and then pass on the weighted output. The input in this context can be some information about a galaxy, a section of an image or an underlying parameter estimation. These nodes all interconnect through many different layers where they all perform operations on some input. These operations all then come together in a final classification layer at the end which outputs either a formal classification or some value that can be mapped to a classification by a user.

A formal classification can be made in this way by training the neural network to recognise certain features of an image. Training a neural network in supervised learning involves providing it with a training set of fully annotated data so it can `learn' classifications form examples. To train the neural network iterates through the provided training set and makes classifications upon each input. It then checks how many it got right and how many wrong, and then tweaks the weights applied to each node throughout it. It then makes the classifications again and checks whether it got more right or wrong. By doing this through many epochs of the training set, it gradually tweaks its internal weights such that it should be able to make the same classifications on an unknown dataset and recognise images with similar classifications to the training set. Thus, it is important that the training set is large and is representative of the full parameters space a user wishes to classify.

In morphology classification, the workhorse neural network is the convolutional neural network (CNN). A CNN takes an image and breaks it down into smaller subsections in a convolutional layer. These smaller subsections - essentially arrays directly from the image - are passed through normal neural network operations and activations and are altered by the weights at each node. Depending on the number of layers in the CNN, the image may be sub-divided, convolved and weighted many times before being fed into a classification step which outputs the final classification. The real power of a CNN such as described is that it is able to recognise features in an images. For instance, spiral arms, tidal features or ellipticals. 

However, using a pure CNN for the purposes of identifying interacting galaxies still introduces the same problems as using visual identification. While it is able to recognise the morphology of interacting galaxies, or recognise two galaxies that are close together it cannot discern this from close pairs. Thus, many works have been done using CNN to identify interacting galaxies and have then had conduct large-scale de-contamination of their final samples. Often, doing so leads to reducing potential sizes of interacting galaxy samples by 50\% - 75\%, with the de-contamination processes requiring visual inspection or further ancillary data which is time consuming to conduct or attain.

New methods combining machine learning with something else have come about in order to overcome this issue. Some works have turned to using the underlying observational parameters of the galaxies to define them \citep{2023ApJ...958...96R}. Others have turned to estimating ancillary parameters in different ways in an attempt to reduce initial contamination. Many works initially attempted to train CNNs on simulated data; where the training set could be controlled and composed of the entire dynamical time of interaction and be completely empty of contamination. However, when applied to observational data it was found that the accuracy of such models completely breaks down \citep{2019MNRAS.490.5390B, 2020A&C....3200390C}.

Thus, the identification of interacting and merging galaxies in an automated fashion remains an open question. Spectroscopic or photometric information for many potential systems does not exist, and therefore cannot be discerned from close pairs. In this work, we present a new way of finding interacting galaxies using a combination of representation learning \citep{2022MNRAS.513.1581W} and visual classification.

\section{This Thesis}
\noindent The primary focus of this thesis is on using tidal features to identify and constrain the relationship between galaxy interaction and evolution. The list of unresolved questions about galaxy interaction is long and complex, and we will focus on the following questions here:

\begin{itemize}
	\item What do we require to improve our understanding of interaction?
	\item How do we better identify interacting galaxies?
	\item What is the relationship between numerous galactic processes and galactic parameters?
	\item From existing samples and studies, how do we better constrain our understanding on the effects of interaction?
\end{itemize}

To fully answer these questions about galaxy interaction, we require three things: first, the tools to find and create large statistically significant samples of interacting galaxies; second, available ancillary data of these systems to make inferences between the underlying processes in interaction and their parameters; and thirdly, the tools to betteer constrain these relationships. In this thesis, detail two pipelines which achive these goals. First, in Chapter \ref{chapter2}, we describe a pipeline for creating the largest interacting galaxy catalogue to date from combining morphological classification with novel methods of data extraction and analysis. We will then use this catalogue, in Chapter \ref{chapter3}, with cross matches to existing ancillary catalogues and conduct our own inferences about the relationship between galactic parameters and underlying processes. However, this will also show the limitations of only using observations and the requirements for constraint by simulations. Finally, in Chapter \ref{chapter4},  we describe a pipeline that will conduct such constraint in the context of Bayesian statistics and galactic parameterisation. We will describe the results we have found by applying it to a small, well constrained, interacting galaxy sample and then describe the limitations of the approach in terms of computational efficiency. Finally, in Chapter \ref{chapter:conclusion} we summarise our results, describe them in the context of current works and speculate about the future work to advance them.