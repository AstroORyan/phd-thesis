\chapter{Introduction}
% Introduce Galaxy Interaction, why is it important?
In the context of large scale structure of the universe, galaxies represent a fundamental building block of matter. Through cosmic time galaxies evolve \citep{2005Natur.435..629S}, this evolution leading to the local universe as we see it today. A key part galaxy evolution is that of mutual interactions between them. Mutual interaction has a number of influences on galaxies, from morphological distortion, to mass transfer, to increases in star formation, to merging where the two (or more) galaxies coalesce into a single system. Our best cosmological model, $\Lambda$-Cold Dark Matter, dictates that matter - ergo, galaxies - assembled hierarchically \citep{1978MNRAS.183..341W, 1991ApJ...379...52W}. Therefore, we must understand the effects of mutual interaction to not just understand galaxy evolution but to better understand our theories of cosmology.

We have come to understand that interaction has multiple impacts on the evolution of galaxies. The first, and most obvious effect, is that it leads to the distortion of the disks involved and the formation of dinstinct tidal features. Since the existence of such features being related to interaction was hypothesised to proved \citep{Toomre_72} they are the best morphological indication that two galaxies are interacting. Often, we attempt to only classify interacting galaxies from their morphology. To truly know if two or more systems are interacting, we require redshift information to ascertain their 3D distances from each other. However, spectroscopic information across the many million close galaxy pairs we know about is limited.

Mutual interaction has further effects on galaxies than simply morphological disturbance, although these remain poorly understood. It is widely debated about the role of interaction in nuclear ignition and enhancement in star formation of each system. While the former may not happen at all, the latter is generally accepted to occur while the specific strength and impact on the system remains highly debated. If such processes occur strongly, they can completely change the system that we are observing. Interaction has also been proved to have an impact on the gas reserves in a galaxy, often leading to the complete quenching of the system post-interaction. However, interaction has also been shown to rejuvenate the galaxies involved, and bring them out of a quenched state and into a star forming state. This is highly dependent on the specifics of the interaction. Finally, the role of environment in galaxy interaction is also a highly researched area.

The primary focus of this thesis is on the role of tidal distortion in formation of tidal features for the purposes of system identification and constraint. However, in this process, we must also keep an account of the other underlying processes occuring during interaction. Fully understanding them, and modelling them if necessary, is key to accurately inferring and linking galactic parameters to physical processes. The list of unresolved questions about galaxy interaction is long and complex. The focus of this thesis is in providing the tools to understand galaxy interaction and answer some of the fundamental questions regarding it. These questions include:

\begin{itemize}
	\item What do we require to improve our understanding of interaction?
	\item How do we better identify interacting galaxies?
	\item What is the relationship between numerous galactic processes and galactic parameters?
	\item From existing samples and studies, how do we better constrain our understanding on the effects of interaction?
\end{itemize}

To fully answer these questions about galaxy interaction, we require three things: first, the tools to find and create large statistically significant samples of interacting galaxies; second, available ancillary data of these systems to make inferences between the underlying processes in interaction and their parameters; and thirdly, the tools to betteer constrain these relationships. In this thesis, detail two pipelines which achive these goals. First,  In chapter 2, I will describe a pipeline for creating the largest interacting galaxy catalogue to date from combining morphological classification with novel methods of data extraction and analysis. I will then use this catalogue, in chapter 3, with cross matches to existing ancillary catalogues and make my own inferences about the relationship between galactic parameters and underlying processes. However, this will also show the limitations of only using observations and the requirements for constraint by simulations. Finally, in chapter 5,  I describe a pipeline that will conduct such constraint in the context of Bayesian statistics and galactic parameterisation. I will describe the results we have found by applying it to a small, well constrained, interacting galaxy sample and then describe the limitations of the approach in terms of computational efficiency. However, for now, it would be prudent to begin this discussion by exploring the definition of  a galaxy, and outlining the major work already conducted into linking galaxy interaction with galaxy evolution.

\section{WHAT IS A GALAXY?}
\noindent For the purposes of this work, a galaxy will be the smallest unit of mass we will be considering. A galaxy is an extended system of stars, gas and dust which all orbits around a supermassive black hole at its centre. All of this baryonic matter makes up the luminous, or visible, matter of the galaxy. This matter has two distinct components: the galactic bulge and the galactic disk. The galactic bulge lies at the centre of the system. It is an approximately spherical component which extends out to about a third of the radius of the galaxy. Galaxies will always have a bulge component, although it might be so small that the galaxy appears to only have a disk. The bulge is often made of an older population of stars, with high peculiar motions in their orbit. The galactic disk is then a very thin, extended region which orbits around the galactic bulge. This is where the majority of the galaxy's gas and dust reside. There are many younger stars throughout it as well, and this extends out to a radii many times that of the galactic bulge. Both of these are embedded in a dark matter halo.

The dark matter halo is many times the radius of the luminous components of the galaxy. This halo is also many times the mass of baryonic components of the galaxy, and makes up the majority of the mass of a galaxy. We approximate each of these components with different potential models. For the galactic bulge, we utilise a Plummer sphere \citep{1911MNRAS..71..460P}. For the galactic disk, we use an exponential disk model. For the dark matter halo, we utilise a cored Navarro-Frenk-White model \citep{1997ApJ...490..493N}.

For much of the history of extragalactic astrophysics, the idea that galaxies could interact and merge was thought to be very inprobable. The small radii of the luminous matter in galaxies, and their calculated cross sections, led astronomers to believe that the probability of an interaction was close to 0. However, with the development of the idea of the dark matter halo, with its radius many times that of the luminous matter, it was realised that the probability of galactic systems interacting and merging was within the realm of possibility. And, therefore, over time the idea that galaxy interaction and merging would have a significant impact on galaxy evolution was developed. In fact, the merger rates of galaxies through cosmic time is one of the most well studied concepts of our cosmological models with observations and simulations.

\section{GALAXY ASSEMBLY ACROSS COSMOLOGICAL TIME}

\section{PROPERTIES OF GALAXIES}

\section{GALAXY MORPHOLOGY}

\section{EVOLUTION OF MORPHOLOGY OVER COSMIC TIME}

\section{CATEGORISATION OF MERGERS}

\section{EFFECTS OF GALAXY INTERACTION}
% Here, present examples of interacting galaxies and the context in which we're talking about them.
Initially, at the discovery of galaxies and they were assumed to be island universes \citep{Hubble paper}. The idea of two or more galaxies interacting and the effect that it would have had upon galaxy formation and evolution was thought to be minute \citep{Found this info in Chapter 8 of Binne and Tremaine. Need to find a better source!}. This assumption was not entirely without merit for the time. By calculating the chance of a galactic encounter from the galactic cross section measured via the galactic radius of stellar material and comparing it to the vast distances between galaxies, it seemed highly improbable that two systems would collide. However, with updated cosmological theories and the introduction of a model where the stellar component of galaxy is embedded in a dark matter halo \citep{Which paper introduced the idea of a galactic halo?}, the size of galaxies and their cross section increased. This dramatically increased the probability of two galaxies interacting in the universe, and with the theory that may of the tidal features found were formed by tidal interaction, a new field of study was born. 

We now understand that galaxy interaction plays an important, and fundamental role in the evolution of galaxies. As stated previously, they have multiple affects upon the systems undergoing the interaction. While the specifics of these effects lies heavily on the underlying parameters and type of interaction that is occurring, the driving force underlying them all is graviational. Therefore, the mass of the systems involved is of great importance to the resultant system we might observe. If the two systems that interact are at vastly different masses - a so-called minor interaction - then the less massive galaxy will be highly perturbed, whilst the more massive galaxy will be relatively intact. If, however, the two systems are of a relatively similar mass both systems will be highly perturbed to the point where they may even be destroyed - a so-called major interaction. A much more difficult type of interaction to observe is a micro-interaction. This is where one system has an almost insignificant mass when compared to the system it is interacting with. In this scenario, the least massive system is always almost entirely destroyed by the more massive system. Specific criteria are often placed upon the mass ratios - the mass of one galaxy when compared to the other - to define these three categories of interaction. For a major interaction the mass ratio must be close to 1:1, in a minor interaction it must be close to 1:10 and in a micro interaction it must be less than 1:10 \citep{A paper that states what the accepted merger ratios are?}. 

However, it is also known that other aspects of the interaction also influence the outcome of the final system. Another that has a direct impact on the output morphology of the system is the velocities and orientations involved in the interaction \citep{Some review paper, perhaps that states that these parameters are important}. A slower interaction with a higher impact parameter (i.e. one where the point of closest approach is small) leads to a more violent interaction with highly perturbed morphology. These slower interactions often lead to the two or more systems becoming intertwined, with multiple passages and merging being the final state of the system. This often occurs due to dynamical friction in the interaction, where heating of the two disks involved and transfer of momentum in the inelastic interaction leads to the two systems spiralling into one another. Again, such an outcome is directly related to the masses of the systems involved as well as their relative velocities and their orientations. 

Many underlying physical processes can also be started or enhanced due to interaction, and the effects of gravity upon the galaxies involved. If the systems involved are gas-rich - a so-called wet merger - then the star formation in the involved systems can be significantly enhanced. It has often been found that galaxies in an interaction or in the post-interaction state are starbursting \citep{Paper on starbursting in interaction}, with their star formation rates many times what would expected of a galaxy of their mass. This increase in star formation then leads to quenching of the system. In the same way, a system that is already quenched can undergo rejuvenation of its gas, and begin star forming after a wet merger. The increased movement in gas throughout the system can also lead to the activation of nuclear activity in the core, and completely alter what we observe of a galaxy. On the other hand, in a dry merger, no increase in star formation is expected and the discussed effects are often surpressed.

So, with these underlying processes in mind, how do they actually occur? What is happening within galaxies that leads to interaction having such a profound effect upon their observed parameters. To understand this, we must go back to first principles and explore how these processes would normally occur and then extrapolate this out the extreme conditions of interaction. I will explore how these conditions relate directly to the underlying parameters of interaction, and the observational effects each result has on a galactic system.

% For each of these sections below, I want to do a deep dive on how each process is driven. Will maybe mention simulations throughout, but we shall see?
\subsection{Morphological Distortion: Tidal Features}
% Introduction here: What do I mean by tidal distortion? What are we actually looking at?
\noindent The clearest example of two galactic systems interacting is the existence of tidal features. These tidal features are results of severe morphological distortion due to tidal forces an interaction. An excellent example which has been very highly studied is NGC 4038 / 4039, also known as the Antennae galaxy. This system is a major interacting pair, which are likely continuing into a merger, with long tidal tails extending far from the interacting cores. The first works attempting to simulate the outcomes of galaxy interaction focused on this system \citep{Works like TandT which looked at this}. The full extent of the tidal tails system was not revealed until deeper observational data became available in 2004 \citep{Citation from BandM}, and then could be accounted for in the simulators modelling.

This is a common problem with information on the tidal features which form in interaction. They often lie in the low surface brightness regime, at the very deepest depths that our telescopes can observe. This often makes investigating any system but the most massive, brightest systems difficult. A classic example of the limitations of our telescopes has been the case of NGC 5907, where the morphology of the stellar stream about it (likely created by a minor interaction) has been under significant scrutiny in recent years \citep{If this paragraph remains here, cite the different NGC 5907 papers}. With the changing morphology of the stellar stream, so to does our understanding of the dynamics of the system as well as estimates of the progenitor galaxy that formed the stellar stream. 

Thus, understanding the dynamics at play in an interaction, and the underlying physics of the different tidal features which form, is of paramount importance to understanding galaxy interaction. Gravity is the driving force of the changing morphology of two systems in an interacting system. Galaxies are composed of three components: a stellar component, a gas component and a dark matter component. In simulations, the stellar and dark matter components are treated as collisionless. Focusing on the stellar component, this is not an unfair assumption when considering the very large distances involved in an interaction to the cross section of a typical stars within each galaxy. Due to this difference, the probability of stars actually colliding is actually very small. However, the distribution of stars is often radically altered by interaction. The gravitational field of each system perturbs the ordered orbits of the stars within each system, transferring energy into them, and adding random motion to stellar orbits \citep{Holmberg 1941 or Alladin 1965}. This is an inelastic collision where as an interaction progresses, and more energy is transfered into the ordered motion of the stars, that the two galaxies merge into a single system. This final result is called a merger remnant. However, if the two systems have the required velocity, they will escape each others potential well and simply move past each other.

% Need to add a paragraph here bringining the stellar distribution back around to tidal features.
This transfer of energy, and disruption of the stellar disk, due to gravity is called dynamical friction. Dynamical friction acts to decelerate the stars in both galaxies, and act as a frictional drag. This force is proportional to the mass squared of the galactic system producing the perturbation. Thus, the density of stars ahead of the moving galaxy becomes less than the density behind it as the stars are decelerated from their original ordered motion. This drag acts to strip the outer regions of the galaxy of stars into tidal bridges between the systems, followed by becoming large tidal tails behind the galaxy. If the two galaxies have the velocity to escape each others potentials, these tidal tails can often coalesce themselves to form tidal dwarfs orbiting the galaxies originally involved in the interaction \citep{Paper about tidal dwarfs}. The tidal tails can also fall onto the secondary galaxy in the interaction, and act as mass transfer between the two systems.

% What is the 'experience' with stars? What about gas?
Thus, while the stellar component of the galaxy is often a simple calculation of the forces on each star by gravity, the same is not true of the gas component. The gas component is made up of multiple much larger clouds of gas within each galaxy that can be approximated as a fluid. If two gas clouds from different galaxies were to collide with each other - a more probable outcome than with two stars - then the shocks driven through the clouds would cause it to very quickly fragment and the formation of stars to occur \citep{Papers on fragmentation of gas in interaction, paper on star formation in star fragments}. These newly formed stars would then heat and ionise the surrounding gas, shocking it again when they go supernova and beginning the process all over again. The motion and position of the gas clouds within the galaxy itself will also be highly affected. Depending on the orientation of such a collision, or the external torques upon the gas cloud from the influence of the two gravitational potentials, the gas cloud may lose so much angular momentum it begins to move into the inner regions of the galaxy \citep{Need a paper which talks about torques on gas in a system and the feedback involved here.}. This increase in the galactic core would then begin the process of star formation again. 

% Finishing talking about gas here allows us to easily follow up and talk about the star formation enhancement next...
This process, by which the star formation in interacting galaxies increases dramatically compared to non-interacting counterparts, is called a starburst and interaction is just one mechanism by which it can occur. A starburst is defined as when the galaxy is forming a mass of stars per year which is significantly higher when compared to what is expected from a galaxy of its mass. This is observed to occur in some, but not all, interactions \citep{Review paper on starbursts in galaxies}. So, we will now discuss the fundamental process of star formation, and how the above described process can lead to a significant enhancement in star formation during an interaction or merger.


\subsection{Star Formation Enhancement} 
\noindent The star formation rate (SFR) of a galaxy is governed, primarily, by its size and the total mass of the gas within it. This is not completely unexpected, as the dominant influence on star formation is the gas density. An excellent way to estimate the SFR in a galaxy is by the Kennicutt-Schmidt Law \citep{Kennicutt-Schmidt Law Paper}

\subsection{Nuclear Activation}
\subsection{Quenching of Galactic Systems}
\subsection{The Role of Environment}